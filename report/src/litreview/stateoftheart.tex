\section{Current State of the Art}
\label{sec:lit:sota}

A comprehensive overview of the current state of the art for combined compressor process and anti-surge control is given by Fausel et al. in \cite{Fausel2010} where they describe the control approach taken by the Compressor Controls Corporation.
Control is performed in the frequency domain using two dedicated process and anti-surge control loops, whose interactions are offset by loop decoupling elements.
An overview of the approach is shown in \fig{lit:sota-diagram}.

\begin{figure}
  Diagram of conventional control approach
  \caption{}
  \label{fig:lit:sota-diagram}
\end{figure}

The main characteristics of the approach are as follows:

\begin{enumerate}
  \item Dedicated process and anti-surge controllers, acting on the driver torque (or gas turbine speed, etc.) and on the recycle valve opening, respectively.
  \item Loop-decoupling components to offset interactions between the process and anti-surge controllers, and between coupled compressors (e.g. connected in series or in parallel).
  \item An anti-surge controller with a PI response to maintain operation on the surge control line.
  \item An additional, open-loop term in the anti-surge controller activated when the surge distance drops below a certain threshold. 
    This term is modulated by the rate of change in the surge distance -- a larger disturbance causing a faster move towards surge triggers a larger open-loop response. 
    The open-loop response then decays exponentially until full control is returned to the PI controller.
    \footnote{This open-loop response can also be triggered stepwise. 
      In this case, the open-loop term is applied for a pre-defined length of time after which the controller checks if the surge distance has increased above the safety threshold. 
      If it has, the term decays exponentially as usual. 
    If the surge distance is still below the threshold, the magnitude of the open-loop term is increased and the process is repeated until the compressor is pulled away from surge.}
\end{enumerate}

The disadvantage with respect to the multi-variable MPC approach considered here is that the two control loops are independent, so in compressors with an electric driver, for example, the driver torque cannot be used to regulate the surge distance and pull the compressor away from surge on a much faster time scale than that achieved using the recycle valve alone.

